\documentclass[twocolumn]{revtex4}
\usepackage{color}
\newcommand{\chg}[1]{\textcolor{green}{#1}}
\begin{document}
\section{Conclusion}
Measured infrared (IR) reflectivity spectra have been reported by several groups, and
\chg{Reaney} et al. \chg{compared} IR spectra from PMN with spectra from
Ba(Os$_{1/2}$Mg$_{1/2}$)B$_{3}$~(BON) and the RFE systems \chg{like}
Pb(Os$_{1/2}$Mg$_{1/2}$)B$_{3}$~ (PON); Pb(In$_{1/2}$Mg$_{1/2}$)B$_{3}$ (PIB); three
Pb(Sc$_{1/2}$Ta$_{1/2}$)B$_{3}$~ (PST) samples (PST-O, PST-D and PST-DV; ordered,
disordered and disordered with Pb-vacancies, respectively). Comparisons between the IR
spectra and electron diffraction data clearly indicates a correlation between chemical
disordering and the depth of a well at $\sim 900 \mathrm{cm}^{-1}$~ (deeper well
$\Rightarrow$~ larger order parameter) in a broad peak that occurs in the frequency
range $\sim 950-1000\mathrm{cm}^{-1}$. The Burns and Dacol spectra exhibit a relatively
deep $\sim 700\mathrm{cm}^{-1}$-well, but in other IR spectra
 this well is absent or very shallow. Burns and Dacol
 interpreted their data as indicating that PMN is either \emph{homogeneous},
or has \emph{large} (macroscopic) heterogeneities; i.e. that it is homogeneous at the
length scale sampled by IR spectroscopy. They also interpreted their three-peak
observation as "three-mode behaviour" which derives from the presence of ions with four
different masses on B-sites. Apparently, however, the presence of two distinct peaks in
this frequency range is indicative of cation ordering (not disorder) that creates two
distinct B-sites, B' and B'', at the IR-sampled length scale.
\end{document}
